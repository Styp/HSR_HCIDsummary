\documentclass[%
	%twocolumn,
	pdftex,%              PDFTex verwenden da wir ausschliesslich ein PDF erzeugen.
	a4paper,%             Wir verwenden A4 Papier.
	landscape,%						Seite - Landscape
	ngerman,
	oneside,%             Einseitiger Druck.
	6pt,%                 Grosse Schrift, besser geeignet f�r A4.
	halfparskip,%         Halbe Zeile Abstand zwischen Abs�tzen.
]{scrbook}



\usepackage[utf8]{inputenc}

\usepackage{amssymb}
\usepackage{amsfonts}
\usepackage{verbatim}
\newcommand{\changefont}[3]{
\fontfamily{#1} \fontseries{#2} \fontshape{#3} \selectfont}

\usepackage[none]{hyphenat}
\sloppy

\usepackage{graphicx}

\usepackage[ngerman]{babel}
%\usepackage{a4wide}
\usepackage{multicol}
%\usepackage{epsfig}
% um eps-einzubinden
\usepackage[landscape]{geometry}

\usepackage{array}

\usepackage[fleqn]{amsmath}
%\usepackage{marvosym}
%\uspackage{amsopn}

\usepackage{color}
\usepackage{tabularx}

\usepackage{fancyvrb} %\usepackage{fancybox}

\usepackage{lmodern}

\usepackage{framed}

\usepackage{listings}             % Include the listings-package

\definecolor{mygreen}{rgb}{0,0.6,0}
\definecolor{mymauve}{rgb}{0.58,0,0.82}
\definecolor{lightgrey}{gray}{0.6}


\lstset{ %
  backgroundcolor=\color{white},   % choose the background 
  basicstyle=\linespread{0.9}\scriptsize,
  language=Java,
  keywordstyle=\color{blue},       % keyword style
  commentstyle=\color{mygreen},       % keyword style
  stringstyle=\color{mymauve},
  frame=single,
  framerule=1px,
  rulecolor=\color{spec_gray}, 
  aboveskip=\smallskipamount, 
  belowskip=\smallskipamount,       % 
  tabsize=2,	                   % sets default tabsize to 2 
}

%\usepackage[retainorgcmds]{IEEEtrantools} %IEEEeqnarray

\usepackage{xfrac} %f"ur sch"one 1/2 etc. Br"uche: \sfrac{1}{2}
\usepackage{cancel} %Durchstreichen in amsmath: \cance{x} oder \cancelto{0}{x}

\definecolor{spec_gray}{gray}{0.5}
\definecolor{spec_lgray}{gray}{0.65}
\definecolor{spec_blue}{rgb}{0,0.37,1}
\definecolor{spec_red}{rgb}{1,0.1,0.1}
\definecolor{spec_llgray}{gray}{0.9}

\newenvironment{bspbox}{%
  \def\FrameCommand{\fboxrule 1pt \colorbox{spec_llgray}}%
  \MakeFramed {\advance\hsize-\width \FrameRestore}}%
 {\endMakeFramed} 
 
% cp850 fr DOS, ansinew fr Windows statt latin1
%\renewcommand{\familydefault}{phv}
% setzt Helvetica (sieht aus wie Arial und sieht auch nach dvi2pdf noch gut aus)

\newcommand{\bsp}[1]{\vspace{-1.5mm}\begin{bspbox}\textbf{Bsp.:}  #1\end{bspbox}\vspace{-1mm}}

\newenvironment{mainbox}{%
  \def\FrameCommand{\fboxrule 1px \fcolorbox{black}{spec_blue}}%
  \MakeFramed {\advance\hsize-\width \FrameRestore}}%
 {\endMakeFramed}
 
\newenvironment{subbox}{%
  \def\FrameCommand{\fboxrule 1px \fcolorbox{black}{spec_gray}}%
  \MakeFramed {\advance\hsize-\width \FrameRestore}}%
 {\endMakeFramed}

\newenvironment{subsubbox}{%
  \def\FrameCommand{\fboxrule 1px \fcolorbox{black}{spec_lgray}}%
  \MakeFramed {\advance\hsize-\width \FrameRestore}}%
 {\endMakeFramed}
 
\newenvironment{titlebox}{%
  \def\FrameCommand{\fboxrule 1pt \fcolorbox{black}{black}}%
  \MakeFramed {\advance\hsize-\width \FrameRestore}}%
 {\endMakeFramed} 
 
% cp850 fr DOS, ansinew fr Windows statt latin1
%\renewcommand{\familydefault}{phv}
% setzt Helvetica (sieht aus wie Arial und sieht auch nach dvi2pdf noch gut aus)

\newcommand{\maintopic}[1]{\setcounter{subtopicenum}{0}\setcounter{subsubtopicenum}{0}\vspace{-4px}\begin{mainbox}\textcolor{white}{\textbf{\large{\stepcounter{maintopicenum}\Roman{maintopicenum}. #1}}}\end{mainbox}\vspace{-4px}}

\newcommand{\subtopic}[1]{\setcounter{subsubtopicenum}{0}\vspace{-4px}\begin{subbox}\textcolor{white}{\textbf{\stepcounter{subtopicenum}\Roman{maintopicenum}.\arabic{subtopicenum} #1}}\end{subbox}\vspace{-4px}}

\newcommand{\subsubtopic}[1]{\vspace{-3px}\begin{subsubbox}\textcolor{white}{\textbf{\stepcounter{subsubtopicenum}\Roman{maintopicenum}.\arabic{subtopicenum}.\arabic{subsubtopicenum} #1}}\end{subsubbox}\vspace{-3px}}

\newcommand{\titletopic}[1]{\vspace{0px}\begin{titlebox}\textcolor{red}{\textbf{#1}}\end{titlebox}\vspace{-3px}}

\newcommand{\vect}[1]{\mathbf{#1}}
\newcommand{\arcsinh}{\operatorname{arcsinh}}
\newcommand{\arccosh}{\operatorname{arccosh}}
\newcommand{\artanh}{\operatorname{artanh}}
\newcommand{\grad}{\operatorname{grad}}
\newcommand{\divergenz}{\operatorname{div}}
\newcommand{\rot}{\operatorname{rot}}
\newcommand{\D}{\,\textrm{d}}
\newcommand{\spec}{\operatorname{spec}}

\renewcommand\arraystretch{1.5}



\CustomVerbatimEnvironment{myverbatim}{Verbatim}{fontsize=\footnotesize,baselinestretch=0.8,numbers=none,showspaces=false,frame=single,rulecolor=\color{spec_gray}}     

\newenvironment{tight-itemize}
{ \begin{itemize}
    \setlength{\itemsep}{0px}
    \setlength{\parskip}{0px}
    \setlength{\parsep}{0px}  }
{ \end{itemize}                  } 

\oddsidemargin-2.2cm
\evensidemargin-2.2cm
\textwidth29cm
\headheight0cm
\topmargin-2.8cm
\textheight20cm
\parskip0cm
\parindent0cm

\parskip-0.5mm


\pagestyle{empty}
%\pagenumbering{arabic}

\begin{document}\changefont{cmss}{m}{n}

\newcounter{maintopicenum}
\newcounter{subtopicenum}
\newcounter{subsubtopicenum}


\begin{multicols}{4}[][-5pt]

\setlength{\abovedisplayshortskip}{-0px}
\setlength{\abovedisplayskip}{-0px}
\setlength{\belowdisplayskip}{-0px} %Mathmode-Commands zum Platz sparen
\setlength{\arraycolsep}{2mm}

\setlength{\topsep}{4px} %Platz sparen bei Titeln
%\setlength{\parsep}{-20mm}

\raggedbottom %whitespace am Ende der letzten Seite statt verteilen des Inhalts auf ben"otigte Seiten



\titletopic{ParProg ZF \qquad \today
\\ Styp \\
HSR FS16 \qquad Prof. Markus Stolze \& al.}
%\qquad \LaTeX{} Design Martin Stypinski
\maintopic{Begriffe}
\begin{tight-itemize}
	\item{\textbf{Prozess} - Schwergewichtsprozess \\
	Parallel laufende Programm-Instanz im System, Eigener Adressraum pro Prozess}
	\item{\textbf{Thread} - Leichtgewichtsprozess \\
	Parallele Ablaufsequenz innerhalb eines Programms, Teilen gleichen Adressraum im Prozess}
	\item{\textbf{Kontextwechsel: Synchron}	Warten auf Bedingung}
	\item{\textbf{Kontextwechsel: Asynchron} Nach gewisser Zeit soll der Thread den Prozessor freigeben, Verhindere, dass ein Thread dauerhaft Prozessor belegt}
	\item{\textbf{Multi-Tasking: Kooperativ} Scheduler kann den Thread nicht unterbrechen, Kontextwechsel wird asynchron initiiert.}
	\item{\textbf{Multi-Tasking: Preemptiv}
	Scheduler kann per Timer-Interrupt unterbrechen (asynchron), Time-Sliced Scheduling.}
\end{tight-itemize}
\maintopic{User Centric Design}
\begin{tight-itemize}
	\item{einen iterativen Prozess durchlaufen}
	\item{in dem die User einbezogen werden}
	\item{Programmierung nach der Konzeption!}
\end{tight-itemize}
\subtopic{Phasen}
\begin{tight-itemize}
	\item{\textbf{Ph1: Analyse} \\
	Einarbeitung, UI Expert Review, Beobachtungen, Interviews, Ist/Soll}
	\item{\textbf{Ph2: Grobkonzeption} \\
	Grobkonzeption, Visualisierung, Prototyp - \textbf{Iterationen}}
	\item{\textbf{Ph3: Grunddesin} \\
	Entwicklung ca. 3 Screens mit grafischem Design}
	\item{\textbf{Ph4: Detailkonzeption} \\
	Interaktiver Prototyp, Detailkonzeption, Usability Tests}
	\item{\textbf{Ph5: Detaildesign \& Dokumentation} \\
	Dokumentation, Detail Desin, Icons, Styleguide}			
\end{tight-itemize}
\subtopic{Vorteile}
\begin{tight-itemize}
	\item{\textbf{kürzere Entwicklungszeiten}}
	\item{\textbf{geringere Betriebskosten} - weniger Komplexität bei gleichem Funktionsumfang}
	\item{\textbf{niedrigere Supportkosten} - weniger Support Anfragen}
	\item{\textbf{verlängerter Lebenszyklus} - weniger Postrelease Änderungen}
	\item{\textbf{höhere Akzeptanz, Produktivität und Rückkehrquoten der User} - weniger Abbruchquoten bei Transaktionen}
	\item{\textbf{mehr Kundenzufriedenheit} - customer experience}
\end{tight-itemize}
\maintopic{Mobile}
\begin{tight-itemize}
	\item{30\% der Arbeitnehmer sind mobil}
\end{tight-itemize}
\subtopic{Mobile Bedürfnisse}
\begin{tight-itemize}
	\item{\textbf{Bedürfnisse}
	\begin{tight-itemize}
	 	\item{\textbf{Microtasking:} Repeating tasks - BEISPIEL?!}
	 	\item{\textbf{I am local:} Finden, Mapdienste}
	 	\item{\textbf{I am bored:} Games, Youtube,...}
	\end{tight-itemize}
	}
	\item{\textbf{Moments}
		\begin{tight-itemize}
		 	\item{\textbf{I-want-to-know:} Wikipedia, Google, etc.}
		 	\item{\textbf{I-want-to-go:} NearMe Search, local business, food,...}
		 	\item{\textbf{I-want-to-do:} Inspration, HowTo, suppporting Vision,...}
		 	\item{\textbf{I-want-to-buy:} Online Shopping, growing convergence on mobile!}
		 	\item{\textbf{Stolze:} Abgedeckt durch 4 Moments}
		\end{tight-itemize}
		}
\end{tight-itemize}
\subtopic{Einschränkungen}
		\begin{tight-itemize}
		 	\item{\textbf{Privacy}}
		 	\item{\textbf{Stress, Ablenkung, Zeitnot} }
		 	\item{\textbf{Kein Strom, Akku leer} }
		 	\item{\textbf{Umgebungsfaktoren} hell, dunkel, laut, leise, kalt, heiss, nass}
		\end{tight-itemize}
\maintopic{Senioren}
		\begin{tight-itemize}
		 	\item{\textbf{Niedrigere Akzeptanz als bei jüngerer Generation}}
		 	\item{\textbf{Verminderte Sehkraft, Farbwiedergabe, Kontraste, Helligkeit} }
		 	\item{\textbf{Schlechtere Akustische Wahrnehmung} }
		 	\item{\textbf{Kürzere Konzentrationsspanne, Aufmerksamkeit} }
		 	\item{\textbf{Motorische Einschränkung}
		\end{tight-itemize}
\maintopic{Jugendliche \& Kinder}
		\begin{tight-itemize}
			\item{Bedürfnisse}
			\item{Einschränkungen}
			\item{TBD!}
		\end{tight-itemize}
\maintopic{Büro-Anwendungen \& Bedürfnisse}
		\begin{tight-itemize}
			\item{Bedürfnisse}
			\item{Einschränkungen}
			\item{TBD!}
		\end{tight-itemize}
\maintopic{
	
\begin{verbatim}





















































































































































































































\end{verbatim}
\end{multicols}

\end{document}